\chapter{About \LaTeX}

\section{Package yang Diperlukan}
Pastikan packages berikut ini telah terinstall pada sistem \LaTeX Anda.

\begin{enumerate}
    \item indentfirst

\item setspace
\item times
\item graphicx
\item latexsym
\item supertabular
\item multirow
\item rotating
\item appendix
\item ifthen
\item nomencl
\item tocloft
\item enumitem
\item caption
\item color
\item listings
\item subfigure
\item url
\item science.sty (scientificpaper)
\item abstract
\end{enumerate}

Sebagian besar dari package tersebut telah terinstall secara default. Apabila ternyata belum terinstall, bisa diunduh dari CTAN.

\section{File yang diubah isinya oleh penulis}
File-file berikut ini diubah oleh penulis sesuai dengan tugas akhir yang dikerjakan
\begin{enumerate}
    \item \texttt{data\_skripsi.tex} \\ berisi data mengenai skripsi, misal judul, nama penguji, nama pembimbing, tanggal ujian, dan sebagainya.
    \item \texttt{persembahan.tex} \\ berisi kepada siapa skripsi ini dipersembahkan (\textit{dedicated}). Halaman persembahan boleh tidak ada. Jika demikian, beri comment pada file \texttt{muka\_skripsi.tex}.
    \item \texttt{motto.tex} \\ berisi moto penulis. Halaman moto boleh tidak ada.  Jika demikian, beri comment pada file \texttt{muka\_skripsi.tex}.
    \item \texttt{prakata.tex} \\ berisi kata pengantar.
    \item \texttt{lambang.tex} \\ berisi input untuk menyusun daftar lambang.
    \item \texttt{abstrakind.tex} \\ berisi abstrak dalam bahasa Indonesia.
    \item \texttt{abstrakeng.tex} \\ berisi abstrak dalam bahasa Inggris.
    \item \texttt{Bab1.tex, Bab2.tex, ..., Bab6.tex} \\ isi dari masing-masing bab.
    \item \texttt{app1.tex, app2.tex, ...} \\ isi dari lampiran. Beri comment pada file \texttt{skripsi.tex} jika tidak terdapat lampiran.
    \item \texttt{pustaka.bib} \\ isi dengan pustaka atau acuan yang digunakan.
\end{enumerate}


\section{Menghasilkan file PDF}
Template ini dibuat secara langsung menggunakan \texttt{PDFLatex}. Jika Anda menggunakan jalur ``biasa'', maka alur yang digunakan adalah:
\begin{equation*}
    latex \longrightarrow dvips \longrightarrow pstopdf
\end{equation*}

Untuk menghasilkan Daftar Lambang, gunakan perintah berikut:
\begin{equation*}
    makeindex \quad skripsi.nlo\quad -s\quad nomencl.ist\quad -o\quad skripsi.nls
\end{equation*}

Jangan lupa untuk me-run latex sekali lagi. Jadi urutan perintahnya adalah
\begin{enumerate}
    \item latex
    \item latex
    \item makeindex 
    \item latex
\end{enumerate}

Jika Anda tidak mengubah Daftar Lambang (dengan mengubah file ``lambang.tex''), maka perintah \texttt{makeindex} tidak perlu dilakukan.




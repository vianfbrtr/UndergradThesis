\Bundengan adalah alat musik tradisional dari Jawa Tengah yang saat ini terancam punah. Salah satu upaya konservasi yang telah dilakukan adalah menampilkan \bundengan pada pementasan publik. Meskipun demikian, upaya ini belum sepenuhnya berhasil karena kualitas bunyi \bundengan yang diterima pendengar kurang optimal. Para insinyur telah berupaya memecahkan masalah ini dengan terlebih dahulu mempelajari bagaimana direktivitas \emph{bundengan}. Meski begitu, para musisi \bundengan belum dapat memahami data direktivitas menggunakan model visualisasi yang sudah ada. Hal ini mengakibatkan upaya pemecahan masalah pementasan \bundengan belum dapat dilakukan karena belum adanya kesepahaman antara para insinyur dan para musisi \bundengan. Dalam penelitian ini penulis merancang dan membangun sebuah sistem visualisasi yang dapat membantu para musisi memahami data direktivitas \bundengan. \par 

Sistem visualisasi yang dibangun berupa perangkat lunak komputer desktop yang dapat menampilkan data pada pengguna secara interaktif. Sistem ini dibangun dengan modul PyQt pada bahasa pemrograman Python. Terdapat dua model visualisasi data pada program ini, yaitu grafik kontur warna dan plot pada diagram polar. Grafik kontur warna merupakan hasil perbaikan dari model visualisasi versi sebelumnya. Dengan sistem ini, skema penyampaian data direktivitas menjadi lebih efektif karena pengguna dapat dengan mudah memilih data sesuai parameter yang diinginkan. Selain itu, kemampuan membandingkan antar frekuensi serta adanya narasi bantuan adalah hal baru yang belum ada di versi sebelumnya. Hasil penilaian responden menunjukkan sistem ini cukup dapat diterima namun mengindikasikan bahwa bimbingan para insinyur ketika para musisi menggunakan sistem tetap diperlukan, khususnya pada penggunaan pertama. \par 

\vspace{0.5cm}
\hspace{-1.2cm}
\textbf{Kata kunci}: \textit{Bundengan}, direktivitas, visualisasi data, PyQt.



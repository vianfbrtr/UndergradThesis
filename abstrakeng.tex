\Bundengan is an endangered traditional musical instrument from Central Java. One of the conservation efforts that have been carried out is displaying \bundengan on stage performances. However, this effort has not been successful because the quality of the \bundengan sound received by the audience is not optimal. Engineers have tried to solve this problem by first studying how the directivity of the \bundengan is. Nevertheless, \bundengan musicians have not been able to understand directivity data using the existing visualization model. This resulted in efforts to solve the problem of \bundengan performances that could not be carried out because there was no understanding between engineers and \bundengan musicians. In this study, the author design and build a visualization system that can help \bundengan musicians to understand the existing directivity data. \par 

The visualization system built by the author is in the form of a desktop computer software program that can display data to the user interactively. This program is built with PyQt module in Python programming language. There are two data visualization models in this program, namely a color contour graph and a plot on polar diagram. The color contour graph is an improvement of visualization model from the previous version. With this system, the directivity data delivery scheme becomes more effective because the user can easily select the data according to the desired parameters. In addition, the ability to compare between frequencies and the presence of help narratives are new things that have not been in the previous version. Assessment results indicate that this system is quite acceptable but indicates that the role of engineers in guiding musicians in using the system are still needed, especially for the first time. \par 

\vspace{0.5cm}
\hspace{-1.2cm}
\textbf{Keywords}: \textit{Bundengan}, directivity, data visualization, PyQt.


Segala puji dan syukur penulis panjatkan kepada Tuhan yang Maha Esa karena atas kehendak-Nya penulis dapat menyelesaikan pengerjaan Tugas Akhir dan penulisan Skripsi yang berjudul "Rancang Bangun Sistem Visualisasi Interaktif Data Direktivitas \emph{Bundengan}". \Bundengan adalah alat musik tradisional yang sangat unik namun saat ini terancam punah. Sejak tahun 2017, para insinyur dari Fakultas Teknik UGM telah berupaya membantu konservasi \bundengan melalui bidang rekayasa. Tujuan besar yang ingin dicapai saat ini adalah terciptanya pementasan \bundengan yang baik supaya \bundengan semakin dikenal di masyarakat. Melalui penelitian ini, penulis ingin berkontribusi dalam upaya melestarikan budaya tradisional. Penulis berharap penelitian ini dapat membantu proses pelestarian \bundengan menjadi satu langkah lebih maju. \par

Penelitian ini tidak akan pernah selesai tanpa bantuan dan dukungan berbagai pihak. Dengan berakhirnya penelitian ini, penulis ingin mengucapkan banyak terima kasih kepada pihak-pihak berikut, di antaranya: \par 

\begin{enumerate}
	\item Dr. Gea Oswah Fatah Parikesit, S.T., M.Sc. dan Dr. Indraswari Kusumaningtyas, S.T., M.Sc. selaku dosen pembimbing penulis. Terima kasih atas kepercayaan dan kesempatan yang diberikan pada penulis untuk mengerjakan penelitian ini. Terima kasih juga untuk segala ilmu yang diberikan selama proses pengerjaan. Penelitian ini adalah pengalaman yang tidak akan pernah penulis lupakan.
	\item (Nama Lengkap Penguji Utama) dan (Nama Lengkap Anggota Penguji) selaku dosen penguji. Terima kasih atas koreksi dan saran yang disampaikan selama ujian pendadaran untuk perbaikan hasil penelitian penulis.
	\item Mas Sa’id selaku narasumber yang bersedia direpotkan penulis selama proses pengerjaan penelitian.
	\item Keluarga penulis yang selalu memberi dukungan dan menjadi alasan kuat penulis menyelesaikan studi.
	\item Sasa yang selalu hadir dan memberi senyuman selama proses pengerjaan penelitian ini.
	\item Teman-teman Tim Akustika Musik: Mas Fairuz, Mas Rachmat, Mba Usi, Astyo, dan Tama.
	\item Teman-teman terdekat penulis yang telah membantu banyak hal selama masa perkuliahan: Arzy, Adnan, Alfi, Basith, Bobby, Diyos, Farrel, Hafidz, Irfan, Jordan, dan Ragil.
\end{enumerate}

Penulis menyadari bahwa penelitian ini masih jauh dari kesempurnaan. Kritik
dan saran dari para pembaca sangatlah berarti untuk perbaikan pekerjaan penulis ke depannya. Terakhir, semoga penelitian ini dapat bermanfaat bagi perkembangan ilmu teknologi dan konservasi kebudayaan. \par 


\begin{flushright}
Yogyakarta, Februari 2022
\end{flushright}
% %
\vspace{0.5cm}
\begin{flushright}
Alvianaka Febriantoro
\end{flushright}
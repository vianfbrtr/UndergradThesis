\chapter{Listing Program}
\singlespacing

%\lstset{language=Delphi,numbers=left, numberstyle=\tiny, stepnumber=1, numbersep=5pt}
\lstset{
language=[90]fortran,
basicstyle=\footnotesize,
numberstyle=\tiny,
numbers=left,
stepnumber=1,
numbersep=10pt,
numberblanklines=false,
columns=flexible,
breaklines=true,
breakindent=10pt
}
\lstset{commentstyle=\textit}

\begin{lstlisting}
program HEAT

use f90_kind
use physconst
use sharevars
implicit none

! locals
integer :: timestep, i, j, ikmax, jkmax, inner_iter, p, q, modval
real(dp) :: start_time, end_time, lambda_g, Nul, Nur, kmax


call cpu_time(start_time)
call read_input
call allocate
call init
call make_grid

call map_nodes

if (restart) call readdata

do timestep=1,No_timesteps
  do inner_iter=1,No_inner_iter

    call calc_tau_drag
    call calc_muturb
    call calc_mfluxes
    call calc_xs
    call calc_weight
    call calc_alpham
    call calc_rad
    call calc_T
    call calc_rhog
    call calc_u
    call calc_v
    call calc_corr
    call calc_mfluxes
    call calc_Pk
    call calc_Gk
    call calc_kturb
    call calc_eps

  enddo

  call relchanges
  call timecopy
  time = time + dt

  modval = mod(timestep, savint)
  if (modval .eq. 0) then
    if (nbr2 == 58) then
      nbr2 = 48
      nbr1 = nbr1 + 1
    endif

    call writeplotmtv
    call writevarplot
    call writedata
    br2 = nbr2+1
  endif

enddo

call cpu_time(end_time)
print *,'#STATISTICS'
print *,'#Runtime: ',end_time-start_time

end program HEAT
\end{lstlisting}

\chapter{KESIMPULAN DAN SARAN} \label{kesimpulan-dan-saran}


%Dikerjakan  \cite{2007} bbb \cite{2009a}. Kalau panjang seperti ini \cite{2007,2008,2003,2005,2009,2008c,2009b, 2009a, 2010}

\section{Kesimpulan}


Kesimpulan merupakan \textbf{rekapitulasi atau rangkuman} dari butir-butir pemikiran utama peneliti. Kesimpulan mencerminkan:
 \begin{enumerate}
 	\item nilai dari penelitian yang dilakukan (sebagai wujud sumbangan orisinal peneliti), dan
 	\item pemahaman peneliti tentang apa yang ditulis.
 \end{enumerate}
 
%\begin{enumerate}
%\item \lipsum[51]
%\item \lipsum[130]
%\item \lipsum[120]
%\end{enumerate}


\section{Saran}

Dalam bagian ini juga bisa disampaikan \textbf{evaluasi} terhadap butir-butir pemikiran utama, misalnya terkait dengan kelemahan metode penelitian yang telah digunakan disertai dengan saran-saran untuk penyempurnaan.

%\begin{enumerate}
%	\item \lipsum[104]
%	\item \lipsum[132]
%	\item \lipsum[115]
%\end{enumerate}

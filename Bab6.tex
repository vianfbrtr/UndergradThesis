\chapter{KESIMPULAN DAN SARAN} \label{kesimpulan-dan-saran}

\section{Kesimpulan}
Setelah melakukan seluruh tahapan proses penelitian, kesimpulan yang dapat diambil dari akhir penelitian ini adalah:
\begin{enumerate}
	\item Sistem visualisasi data direktivitas \bundengan telah berhasil dibangun, yaitu berupa perangkat lunak bernama BundenganDVis pada komputer dengan sistem operasi Windows.
	\item Hasil pengujian kinerja sistem menunjukkan bahwa BundenganDVis cukup dapat diterima oleh calon pengguna, dengan catatan perlu adanya bimbingan dari para insinyur ketika proses penggunaan yang pertama kalinya.
\end{enumerate}


\section{Saran}
Meski tujuan dari penelitian telah tercapai, masih banyak kekurangan yang perlu diperbaiki dalam penelitian ini. Berikut adalah beberapa saran yang dapat diterapkan untuk memperbaiki penelitian ini di masa mendatang:
\begin{enumerate}
	\item Perlu adanya pengujian kinerja sistem pada pengguna yang sesungguhnya (pemusik \emph{bundengan}) dengan pengetahuan sains seminim mungkin.
	\item Perlu adanya penelitian lanjutan untuk mengetahui fitur yang belum terdapat pada sistem saat ini namun diperlukan pengguna dalam pemahaman data direktivitas \bundengan ataupun perancangan pementasan \emph{bundengan}.
	\newpage
	\item Perlu adanya pengembangan sistem dari yang sekedar menampilkan data menjadi sebuah sistem yang otomatis mengolah dan dapat menampilkan visualisasi data secara \emph{real time}.
	\item Perlu adanya pertimbangan untuk mengembangkan BundenganDVis ke dalam bentuk yang lebih cocok di masa yang akan datang.
\end{enumerate}
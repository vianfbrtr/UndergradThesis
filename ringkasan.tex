\documentclass[a4paper,10pt]{article}

\usepackage{packages/science}

\usepackage[latin1]{inputenc}
\usepackage{amsfonts}
\usepackage{amssymb}
\usepackage{amsthm}
\usepackage{amsmath}
\usepackage[bahasa]{packages/babel}
\usepackage[labelsep=period,labelfont=bf, format=hang]{caption}
\usepackage{fontenc}
\usepackage{graphicx}
\usepackage{lipsum}
\usepackage{abstract}
\usepackage{times}
\usepackage{indentfirst}
\usepackage{url}
\urlstyle{rm}
 
\renewcommand\thesection{\Roman{section}.}
\newcommand{\SSS}{\renewcommand{\baselinestretch}{1.5}\tiny \normalsize }
\newcommand{\SDS}{\renewcommand{\baselinestretch}{1}\tiny \normalsize }
\renewcommand{\abstractname}{Intisari}

\textheight 250mm
\textwidth 160mm
\headheight 0pt
\headsep 30pt
\topmargin -0.5cm
\oddsidemargin 4mm
\evensidemargin 4mm
\parindent 5mm
\hoffset -7mm
\voffset -15mm

\title{\textbf{
JUDUL SKRIPSI DALAM BAHASA INDONESIA
}}
\author{
Nama Lengkap Mahasiswa \thanks{mhs@ugm.ac.id}, 
Nama Pembimbing Utama \thanks{pembimbing@ugm.ac.id}, 
Nama Pembimbing Pendamping \thanks{pembimbing2@ugm.ac.id} \\ \\ 
\textit{Jurusan Teknik Fisika Fakultas Teknik} \\ \textit{Universitas Gadjah Mada}}

\date{ }

\begin{document}

 %\twocolumn[   
\maketitle

%\begin{onecolabstract}
\begin{abstract}
    


Lorem ipsum dolor sit amet, consectetuer adipiscing elit. Ut purus elit, vestibulum ut,
placerat ac, adipiscing vitae, felis. Curabitur dictum gravida mauris. Nam arcu libero, nonummy
eget, consectetuer id, vulputate a, magna. Donec vehicula augue eu neque. Pellentesque habitant
morbi tristique senectus et netus et malesuada fames ac turpis egestas.

Mauris ut leo. Cras viverra metus rhoncus sem. Nulla et lectus vestibulum urna fringilla
ultrices. Phasellus eu tellus sit amet tortor gravida placerat. Integer sapien est, iaculis in, pretium
quis, viverra ac, nunc. Praesent eget sem vel leo ultrices bibendum.

Aenean faucibus. Morbi dolor nulla, malesuada eu, pulvinar at, mollis ac, nulla. Curabitur
auctor semper nulla. Donec varius orci eget risus. Duis nibh mi, congue eu, accumsan eleifend,
sagittis quis, diam. Duis eget orci sit amet orci dignissim rutrum.

\vspace{0.5cm}
\textbf{Kata kunci}: katakunci1, katakunci2, katakunci3, katakunci4

%\end{onecolabstract}
%]
\end{abstract}


\begin{body}
\SSS
\section{Pendahuluan}
\lipsum[1-2]

\section{Studi Pustaka}
\lipsum[1-2]

\section{Dasar Teori}
\lipsum[5-6]

\begin{equation}
    \dfrac{Dv}{Dt} = \dfrac{\partial v}{\partial t} + \nabla \cdot \mathbf{uu}
\label{eq:1}
\end{equation}



\section{Pelaksanaan Penelitian}
\lipsum[50-51]
\begin{figurehere}
%\begin{figure}
    \centering
    \includegraphics[width=3cm]{logoUGM.jpg}
    \caption{Contoh penulisan judul gambar dan peletakkan gambar. Gambar harus dilengkapi
dengan informasi yang memadai sehingga mudah ditafsirkan tanpa harus membaca
isi teks (\textit{self-contained}).}
    \label{fig:1}
\end{figurehere}
%\end{figure}

\section{Hasil dan Pembahasan}
\lipsum[76-77]
\begin{tablehere}

%\begin{table}
    

    \caption{Contoh penulisan tabel dan peletakkan tabel.}
    \label{tbl:1}
    \begin{center}
    

\begin{tabular}{|c|c|c|}
\hline
Header 1 & Header 2 & Header 3 \\ 
\hline
Isi & Isi & Isi \\ 
\hline
Isi & Isi & Isi \\
\hline
\end{tabular}
\end{center}

\end{tablehere}
%\end{table}

\section{Kesimpulan dan Saran}
\lipsum[90-91]

\cite{book1, book2, skripsi, tesis, disertasi, kp, article1, proceeding1, penelitian, dokteknama, dokteknonama, majalah1, majalah2, web1, web2, presentasi, diktat,komprivat}

\SDS
\bibliographystyle{skripsi}
\bibliography{ringkasan}
\end{body}
\end{document}
